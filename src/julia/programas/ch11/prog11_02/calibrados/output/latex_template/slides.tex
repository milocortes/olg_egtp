\documentclass[10pt,a4paper]{report}
\usepackage[utf8]{inputenc}
\usepackage[spanish]{babel}
\usepackage{amsmath}
\usepackage{amsfonts}
\usepackage{amssymb}
\usepackage{graphicx}
\usepackage{lscape}
%% Para incluir longtables
\usepackage{longtable}
%% Para incluir item en las celdas de la tabla
\usepackage{booktabs}% http://ctan.org/pkg/booktabs
\newcommand{\tabitem}{~~\llap{\textbullet}~~}
\usepackage[bottom=0.5cm, right=1.5cm, left=1.5cm, top=1.5cm]{geometry}

\sffamily

\begin{document}

\begin{landscape}

\begin{longtable}{p{5cm} p{1cm} p{1cm} p{1cm} p{1cm} p{1cm} p{1cm} p{1cm} p{1cm} p{1cm} p{1cm} p{1cm}}  
%\caption{Agentes del modelo}\label{tab:agentes}\\    %%%%<===
\\[-1.8ex]\hline 
\endhead
\hline \\[-1.8ex] 
\multicolumn{1}{c}{} & \multicolumn{3}{c}{México} & \multicolumn{4}{c}{Chile} &   \multicolumn{4}{c}{Costa Rica} \\
%\cline{2-4}
%\cline{6-7}
\\ 
\multicolumn{1}{c}{} & \multicolumn{2}{c}{Modelo}  & \multicolumn{1}{c}{Observado}  & \multicolumn{2}{c}{Modelo} & \multicolumn{1}{c}{Observado} & \multicolumn{2}{c}{Modelo} & \multicolumn{1}{c}{Observado}\\
\\
\cline{2-3}
\cline{5-6}
\cline{8-9}

\\ 
\multicolumn{1}{c}{} & \multicolumn{1}{c}{\small (1)} & \multicolumn{1}{c}{\small (2)}  & \multicolumn{1}{c}{}  & \multicolumn{1}{c}{\small (1)} & \multicolumn{1}{c}{\small (2)} & \multicolumn{1}{c}{} & \multicolumn{1}{c}{\small (1)} & \multicolumn{1}{c}{\small (2)} & \multicolumn{1}{c}{}\\

\\
\hline \\[-1ex] 

%%++++++++++++++++++++++++++++++++++++++++++++++++++++%%
%%        Indicadores
%%++++++++++++++++++++++++++++++++++++++++++++++++++++%%

Mercado de Bienes (\% PIB)

\par\tabitem Consumo Privado

\par\tabitem Gasto Público

\par\tabitem Inversión

\par ""

Tasas de impuestos (en \%) 

\par\tabitem Consumo 

\par\tabitem Ingreso

\par\tabitem Ingreso medio

\par\tabitem Ingreso máximo

\par\tabitem Ingreso mínimo


\par ""

Ingresos por impuestos (\% PIB) 

\par\tabitem Consumo 

\par\tabitem Ingreso

\par ""

Gobierno (en \% PIB)
\par\tabitem Endeudamiento público 
\par\tabitem Flujo de deuda

\par ""

Sistema de pensiones 
\par\tabitem Tasa de reemplazo
\par\tabitem Pagos a pensiones (\% PIB)

\par ""

Otros
\par\tabitem TC Poblacional (\%)


%%%%%%%%%%%%%%%%%%%%%%%%%%%%%%%%%%%%%%%%%%%%%%%%%%%%%%%%%
%%%%%%%%%%%%%%%%%%%%%%%%%%%%%%%%%%%%%%%%%%%%%%%%%%%%%%%%%
%%++++++++++++++++++++++++++++++++++++++++++++++++++++++%

%               MEXICO

%%+++++++++++++++++++++++++++++++++++++++++++++++++++++++%
%%%%%%%%%%%%%%%%%%%%%%%%%%%%%%%%%%%%%%%%%%%%%%%%%%%%%%%%%
%%%%%%%%%%%%%%%%%%%%%%%%%%%%%%%%%%%%%%%%%%%%%%%%%%%%%%%%%

%%++++++++++++++++++++++++++++++++++++++++++++++++++++%%
%%        Valores Modelo DSOLG México
%%++++++++++++++++++++++++++++++++++++++++++++++++++++%%
&

%%%% Mercado de Bienes
\par ""
% Consumo Privado % PIB 
\par 70.11 
% Gasto Público % PIB 
\par 18.2 
% Inversión % PIB 
\par 11.69 

%%%% Tasas de impuestos
\par ""
\par ""
% Impuesto al Consumo
\par 16.0 
% Impuesto al Ingreso
\par 47.72 
% Impuesto al Ingreso Medio (Observado)
\par ""
% Impuesto al Ingreso Máximo (Observado)
\par ""
% Impuesto al Ingreso Mínimo (Observado)
\par ""
\par ""
\par ""

%%%% Ingresos por impuestos
% Por impuesto al consumo
\par 11.22
% Por impuesto al ingreso
\par 18.04

%%%% Gobierno
\par ""
\par ""
% Endeudamiento Público
\par 44.2
% Costo necesitado para mantener el nivel de deuda constante
\par 11.05

%%%% Sistema de Pensiones
\par ""
\par ""
% Tasa de Reemplazo
\par 0.64
% Pagos a pensiones
\par 4.15

%%%% Otros

\par ""
\par ""
% Tasa de crecimiento poblacional
\par 2.14

%%++++++++++++++++++++++++++++++++++++++++++++++++++++%%
%%        Valores Modelo DSOLG Probabilidades México
%%++++++++++++++++++++++++++++++++++++++++++++++++++++%%
&
%%%% Mercado de Bienes
\par ""
% Consumo Privado % PIB 
\par 62.6 
% Gasto Público % PIB 
\par 18.2 
% Inversión % PIB 
\par 19.19 

%%%% Tasas de impuestos
\par ""
\par ""
% Impuesto al Consumo
\par 16.0 
% Impuesto al Ingreso
\par 36.82 
% Impuesto al Ingreso Medio (Observado)
\par ""
% Impuesto al Ingreso Máximo (Observado)
\par ""
% Impuesto al Ingreso Mínimo (Observado)
\par ""
\par ""
\par ""

%%%% Ingresos por impuestos
% Por impuesto al consumo
\par 10.02
% Por impuesto al ingreso
\par 13.92

%%%% Gobierno
\par ""
\par ""
% Endeudamiento Público
\par 44.2
% Costo necesitado para mantener el nivel de deuda constante
\par 5.73

%%%% Sistema de Pensiones
\par ""
\par ""
% Tasa de Reemplazo
\par 0.64
% Pagos a pensiones
\par 4.96

%%%% Otros

\par ""
\par ""
% Tasa de crecimiento poblacional
\par 2.14



%%++++++++++++++++++++++++++++++++++++++++++++++++++++%%
%%        Valores Observados México
%%++++++++++++++++++++++++++++++++++++++++++++++++++++%%
&  

%%%% Mercado de Bienes
\par ""
% Consumo Privado % PIB 
\par 64.23 
% Gasto Público % PIB 
\par 18.2 
% Inversión % PIB 
\par 21.08 

%%%% Tasas de impuestos
\par ""
\par ""
% Impuesto al Consumo
\par 16.0 
% Impuesto al Ingreso
\par ""
% Impuesto al Ingreso Medio (Observado)
\par 10.2 
% Impuesto al Ingreso Máximo (Observado)
\par 35.0 
% Impuesto al Ingreso Mínimo (Observado)
\par 1.92 
\par ""
\par ""

%%%% Ingresos por impuestos
% Por impuesto al consumo
\par 6.63
% Por impuesto al ingreso
\par 6.01

%%%% Gobierno
\par ""
\par ""
% Endeudamiento Público
\par 44.2
% Costo necesitado para mantener el nivel de deuda constante
\par 5.73

%%%% Sistema de Pensiones
\par ""
\par ""
% Tasa de Reemplazo
\par 0.64
% Pagos a pensiones
\par 2.7

%%%% Otros

\par ""
\par ""
% Tasa de crecimiento poblacional
\par 2.14

%%%%%%%%%%%%%%%%%%%%%%%%%%%%%%%%%%%%%%%%%%%%%%%%%%%%%%%%%
%%%%%%%%%%%%%%%%%%%%%%%%%%%%%%%%%%%%%%%%%%%%%%%%%%%%%%%%%
%%++++++++++++++++++++++++++++++++++++++++++++++++++++++%

%               CHILE

%%+++++++++++++++++++++++++++++++++++++++++++++++++++++++%
%%%%%%%%%%%%%%%%%%%%%%%%%%%%%%%%%%%%%%%%%%%%%%%%%%%%%%%%%
%%%%%%%%%%%%%%%%%%%%%%%%%%%%%%%%%%%%%%%%%%%%%%%%%%%%%%%%%

%%++++++++++++++++++++++++++++++++++++++++++++++++++++%%
%%        Valores Modelo DSOLG Chile
%%++++++++++++++++++++++++++++++++++++++++++++++++++++%%
&

%%%% Mercado de Bienes
\par ""
% Consumo Privado % PIB 
\par 66.33 
% Gasto Público % PIB 
\par 17.07 
% Inversión % PIB 
\par 16.6 

%%%% Tasas de impuestos
\par ""
\par ""
% Impuesto al Consumo
\par 19.0 
% Impuesto al Ingreso
\par 19.55 
% Impuesto al Ingreso Medio (Observado)
\par ""
% Impuesto al Ingreso Máximo (Observado)
\par ""
% Impuesto al Ingreso Mínimo (Observado)
\par ""
\par ""
\par ""

%%%% Ingresos por impuestos
% Por impuesto al consumo
\par 12.6
% Por impuesto al ingreso
\par 8.6

%%%% Gobierno
\par ""
\par ""
% Endeudamiento Público
\par 30.3
% Costo necesitado para mantener el nivel de deuda constante
\par 4.13

%%%% Sistema de Pensiones
\par ""
\par ""
% Tasa de Reemplazo
\par 0.37
% Pagos a pensiones
\par 3.13

%%%% Otros

\par ""
\par ""
% Tasa de crecimiento poblacional
\par 1.74

%%++++++++++++++++++++++++++++++++++++++++++++++++++++%%
%%        Valores Modelo DSOLG Probabilidades Chile
%%++++++++++++++++++++++++++++++++++++++++++++++++++++%%
&
%%%% Mercado de Bienes
\par ""
% Consumo Privado % PIB 
\par 60.09 
% Gasto Público % PIB 
\par 17.07 
% Inversión % PIB 
\par 22.84 

%%%% Tasas de impuestos
\par ""
\par ""
% Impuesto al Consumo
\par 19.0 
% Impuesto al Ingreso
\par 18.6 
% Impuesto al Ingreso Medio (Observado)
\par ""
% Impuesto al Ingreso Máximo (Observado)
\par ""
% Impuesto al Ingreso Mínimo (Observado)
\par ""
\par ""
\par ""

%%%% Ingresos por impuestos
% Por impuesto al consumo
\par 11.42
% Por impuesto al ingreso
\par 8.18

%%%% Gobierno
\par ""
\par ""
% Endeudamiento Público
\par 30.3
% Costo necesitado para mantener el nivel de deuda constante
\par 2.53

%%%% Sistema de Pensiones
\par ""
\par ""
% Tasa de Reemplazo
\par 0.37
% Pagos a pensiones
\par 3.79

%%%% Otros

\par ""
\par ""
% Tasa de crecimiento poblacional
\par 1.74



%%++++++++++++++++++++++++++++++++++++++++++++++++++++%%
%%        Valores Observados Chile
%%++++++++++++++++++++++++++++++++++++++++++++++++++++%%
&  

%%%% Mercado de Bienes
\par ""
% Consumo Privado % PIB 
\par 59.57 
% Gasto Público % PIB 
\par 17.07 
% Inversión % PIB 
\par 24.63 

%%%% Tasas de impuestos
\par ""
\par ""
% Impuesto al Consumo
\par 19.0 
% Impuesto al Ingreso
\par ""
% Impuesto al Ingreso Medio (Observado)
\par 7.0 
% Impuesto al Ingreso Máximo (Observado)
\par 40.0 
% Impuesto al Ingreso Mínimo (Observado)
\par 0.0 
\par ""
\par ""

%%%% Ingresos por impuestos
% Por impuesto al consumo
\par 7.46
% Por impuesto al ingreso
\par 10.2

%%%% Gobierno
\par ""
\par ""
% Endeudamiento Público
\par 30.3
% Costo necesitado para mantener el nivel de deuda constante
\par 2.53

%%%% Sistema de Pensiones
\par ""
\par ""
% Tasa de Reemplazo
\par 0.37
% Pagos a pensiones
\par 2.8

%%%% Otros

\par ""
\par ""
% Tasa de crecimiento poblacional
\par 1.74


%%%%%%%%%%%%%%%%%%%%%%%%%%%%%%%%%%%%%%%%%%%%%%%%%%%%%%%%%
%%%%%%%%%%%%%%%%%%%%%%%%%%%%%%%%%%%%%%%%%%%%%%%%%%%%%%%%%
%%++++++++++++++++++++++++++++++++++++++++++++++++++++++%

%               COSTA RICA

%%+++++++++++++++++++++++++++++++++++++++++++++++++++++++%
%%%%%%%%%%%%%%%%%%%%%%%%%%%%%%%%%%%%%%%%%%%%%%%%%%%%%%%%%
%%%%%%%%%%%%%%%%%%%%%%%%%%%%%%%%%%%%%%%%%%%%%%%%%%%%%%%%%
%%++++++++++++++++++++++++++++++++++++++++++++++++++++%%
%%        Valores Modelo DSOLG Costa Rica
%%++++++++++++++++++++++++++++++++++++++++++++++++++++%%
&

%%%% Mercado de Bienes
\par ""
% Consumo Privado % PIB 
\par 66.42 
% Gasto Público % PIB 
\par 20.47 
% Inversión % PIB 
\par 13.11 

%%%% Tasas de impuestos
\par ""
\par ""
% Impuesto al Consumo
\par 13.0 
% Impuesto al Ingreso
\par 31.28 
% Impuesto al Ingreso Medio (Observado)
\par ""
% Impuesto al Ingreso Máximo (Observado)
\par ""
% Impuesto al Ingreso Mínimo (Observado)
\par ""
\par ""
\par ""

%%%% Ingresos por impuestos
% Por impuesto al consumo
\par 8.63
% Por impuesto al ingreso
\par 18.42

%%%% Gobierno
\par ""
\par ""
% Endeudamiento Público
\par 47.8
% Costo necesitado para mantener el nivel de deuda constante
\par 6.58

%%%% Sistema de Pensiones
\par ""
\par ""
% Tasa de Reemplazo
\par 0.73
% Pagos a pensiones
\par 8.24

%%%% Otros

\par ""
\par ""
% Tasa de crecimiento poblacional
\par 1.79

%%++++++++++++++++++++++++++++++++++++++++++++++++++++%%
%%        Valores Modelo DSOLG Probabilidades Costa Rica
%%++++++++++++++++++++++++++++++++++++++++++++++++++++%%
&
%%%% Mercado de Bienes
\par ""
% Consumo Privado % PIB 
\par 62.92 
% Gasto Público % PIB 
\par 20.47 
% Inversión % PIB 
\par 16.61 

%%%% Tasas de impuestos
\par ""
\par ""
% Impuesto al Consumo
\par 13.0 
% Impuesto al Ingreso
\par 28.6 
% Impuesto al Ingreso Medio (Observado)
\par ""
% Impuesto al Ingreso Máximo (Observado)
\par ""
% Impuesto al Ingreso Mínimo (Observado)
\par ""
\par ""
\par ""

%%%% Ingresos por impuestos
% Por impuesto al consumo
\par 8.18
% Por impuesto al ingreso
\par 16.84

%%%% Gobierno
\par ""
\par ""
% Endeudamiento Público
\par 47.8
% Costo necesitado para mantener el nivel de deuda constante
\par 4.55

%%%% Sistema de Pensiones
\par ""
\par ""
% Tasa de Reemplazo
\par 0.73
% Pagos a pensiones
\par 9.98

%%%% Otros

\par ""
\par ""
% Tasa de crecimiento poblacional
\par 1.79



%%++++++++++++++++++++++++++++++++++++++++++++++++++++%%
%%        Valores Observados Costa Rica
%%++++++++++++++++++++++++++++++++++++++++++++++++++++%%
&  

%%%% Mercado de Bienes
\par ""
% Consumo Privado % PIB 
\par 68.09 
% Gasto Público % PIB 
\par 20.47 
% Inversión % PIB 
\par 17.83 

%%%% Tasas de impuestos
\par ""
\par ""
% Impuesto al Consumo
\par 13.0 
% Impuesto al Ingreso
\par ""
% Impuesto al Ingreso Medio (Observado)
\par 10.5 
% Impuesto al Ingreso Máximo (Observado)
\par 24.6 
% Impuesto al Ingreso Mínimo (Observado)
\par 0.0 
\par ""
\par ""

%%%% Ingresos por impuestos
% Por impuesto al consumo
\par nan
% Por impuesto al ingreso
\par nan

%%%% Gobierno
\par ""
\par ""
% Endeudamiento Público
\par 47.8
% Costo necesitado para mantener el nivel de deuda constante
\par 4.55

%%%% Sistema de Pensiones
\par ""
\par ""
% Tasa de Reemplazo
\par 0.73
% Pagos a pensiones
\par 4.9

%%%% Otros

\par ""
\par ""
% Tasa de crecimiento poblacional
\par 1.79

 \\
\hline 
\hline \\[-1.8ex] 

\end{longtable} 
\end{landscape}


\end{document}